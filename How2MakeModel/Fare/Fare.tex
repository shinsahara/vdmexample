\documentclass[a4paper,10pt]{jsarticle}
%\documentclass[a4paper,10pt]{jsbook}

\usepackage[dvipdfmx]{graphicx, color}
%\usepackage{folha}
\graphicspath{{image/}}

\usepackage{color}
\usepackage{array}
\usepackage{longtable}
\usepackage{alltt}
\usepackage{graphics}
\usepackage{vpp-nms}
%\usepackage{vpp}
\usepackage{makeidx}
\makeindex

\usepackage{colortbl}

\usepackage[dvipdfmx,bookmarks=true,bookmarksnumbered=true,colorlinks,plainpages=true]{hyperref}

%\AtBeginDvi{\special{pdf:tounicode 90ms-RKSJ-UCS2}}
\AtBeginDvi{\special{pdf:tounicode EUC-UCS2}}

\definecolor{covered}{rgb}{0,0,0}      %black
\definecolor{not-covered}{rgb}{1,0,0}  %red

\setcounter{secnumdepth}{6}
\makeatletter
\renewcommand{\paragraph}{\@startsection{paragraph}{4}{\z@}%
  {1.5\Cvs \@plus.5\Cdp \@minus.2\Cdp}%
  {.5\Cvs \@plus.3\Cdp}%
  {\reset@font\normalsize\bfseries}}
\makeatother

\renewcommand{\sf}{\sffamily \color{blue}}

\newcommand{\syou}{\texttt{<}}
\newcommand{\dai}{\texttt{>}}

%\include{Title}

%\pagestyle{empty}
\usepackage{fancyhdr}
\usepackage{lastpage} 
  \pagestyle{fancy} 
   \let\origtitle\title 
  \renewcommand{\title}[1]{\lfoot{#1}\origtitle{#1}}

  \rfoot{\today}
  \rhead{[\ \scshape\oldstylenums{\thepage}\ / %
      \scshape\oldstylenums{\pageref{LastPage}}\ ]}
  \cfoot{}


\begin{document}

% the title page
\title{運賃計算システムVDM++仕様}
\author{佐原 伸\\\\
SCSK Corporation\\
VDM推進担当
}
%\institute{\pgldk \and \chessnl}
\date{\mbox{}}
\maketitle
\begin{abstract}
レコードと列を使った要求仕様の例。

距離から運賃を求めるようにしている。

Makefileを使って、makeコマンドで、モデルの検証とドキュメントのPDFファイル生成を一挙に行うようにしている。

\end{abstract}

%\TaoReport{ガードコマンド・モデル}{\today}{タオベアーズ}{佐原伸}
%\setlength{\baselineskip}{12pt plus .1pt}
%\tolerance 10000
\tableofcontents
%\thispagestyle{empty} 

\section{クラス図}
\index{クラス図}

VDM++仕様のクラス図は以下の通りである。

\begin{figure}[h]
	\centering
	{\includegraphics[width=55zw, keepaspectratio]{image/ClassDiagram1.jpg}}
	\caption{ Class Diagram}
	\label{fig:ClassDiagram}
\end{figure}

%\include{Abstract}
\include{CalcFare.vdmpp}
\include{FareTableDic.vdmpp}
\include{railway_network.vdmpp}
\include{route_search.vdmpp}
\include{route_search_by_dijkstra.vdmpp}
\include{dijkstra.vdmpp}
\include{railway_netwoprk_data.vdmpp}
\include{route_search_testspec.vdmpp}
\include{MyTest.vdmpp}
\include{MyTestCase.vdmpp}

%\begin{thebibliography}{9}
\section{参考文献、索引}
VDM++\cite{CSK2007PP}は、
1970年代中頃にIBMウィーン研究所で開発されたVDM-SL\cite{CSK2007SL}を拡張し、
さらにオブジェクト指向拡張した
\footnote{使用に際しては、(株)CSKシステムズとの契約締結が必要になる。}の形式仕様記述言語である。
\bibliographystyle{jplain}
%\bibliography{/Users/sahara/svnw/sahara}
\bibliography{/Users/sahara/bib/saharaUTF8}
%\bibliography{/Users/ssahara/svnwork/sahara}

%\end{thebibliography}

%\newpage
%\addcontentsline{toc}{section}{Index}
\printindex

\end{document}
