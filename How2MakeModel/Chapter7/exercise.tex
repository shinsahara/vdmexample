本章では、演習問題として以下で説明する図書館システムの例題をあげる。

この図書館システムは、どなたでも基本的な機能を理解でき、
規模も非常に小さいが、
我々が開発するシステムの仕様としての基本的な特徴を持っているからである。

\section {図書館システム}
	\label{LibraryRequirement}
	\index{としょかんしすてむ@図書館システム}
	\index{としょかんしすてむへのようきゅう@図書館システムへの要求}
以下の機能を持つ図書館システムを、VDM++でモデル化する。

\begin{enumerate}
\item 本を図書館の蔵書として追加、削除する。
\item 題名と著者と分野のいずれかで本を検索する。
\item 利用者が、蔵書を借り、返す。
\item 蔵書の追加・削除や、利用者への貸出・返却は職員が行う。
\item 利用者や職員の権限などは考慮しなくてよい。
\item 最大蔵書数は10000、利用者一人への最大貸出冊数は3冊とする。
\end{enumerate}

このシステムを、\ref{How2Intro}章と\ref{How2MakeModel}章で説明した方法でモデル化してみよう。

最初にユースケースレベルの要求仕様を作成する。
	\index{ゆーすけーすれべるのようきゅうしよう@ユースケースレベルの要求仕様}
実行不可能だが静的検証のできる陰仕様を作成し、
次に実行可能で動的検証のできる陽仕様を作成し、回帰テストによりテストせよ。

次に、設計仕様を、VDM++を使ったオブジェクト指向モデルとして作成せよ。

解答例は、\ref{LibraryModel}章に示す。