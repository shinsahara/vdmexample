\section {発端}
鉄道会社Aの特急券予約システムで特急券を2枚予約し、
以下のような「事件」が起こったのが本モデルを書いた理由である。


以下のやりとりは、電話でのやりとりをかなり(主として用語を中心として)整理したものである。
実際には、例えば、鉄道会社Aサポートセンターの担当者は「カード」という言い方をするのだが、
それがクレジットカードの場合と、予約会員証の場合と、ICカードあるいは予約カードの場合があったので、
個々の用語の正確な名前と意味を調べるだけで
かなりの日数と時間を要した問答であった。

ICカードと予約カードは、結局、この問題とは直接は関係なかったので、この後、本稿に登場することはない。

\begin{itemize}
\item おサイフケータイ(鉄道会社B)で特急券予約システム(鉄道会社A)を使っていた
\item ロンロンViewカードが廃止になりアトレクラブViewカードに変更して下さいとの連絡があった
\item 予約できなかったので 鉄道会社Aサポートセンターに電話

	\begin{description}
	\item [客]「特急券を2枚予約しようとしたが、予約できなかったんですが?」
	\item [JR]「カードを変更したら、特急券予約システムを新規に契約して下さい。」
	\item [客]「新規にすると会費がかかるのでは?」
	\item [JR]「はい。」
	\item [客]「カード会社の都合で変更するのに、それはおかしいでしょう?」
	\item [JR]「では、無料にします。」
	\end{description}

\item 特急券に引換えできなかったので 鉄道会社Aサポートセンターに電話

	\begin{description}
	\item [客]「新しい予約はできたんですが、クレジットカード変更前に予約した特急券に引換えようとしたらできないのですが?」
	\item [JR]「予約会員証で引換できるようになったので、それで引換えて下さい。
			暗証番号はクレジットカードのものを使って下さい。」
	\end{description}

\item T駅での問答

	\begin{description}
	\item [客]「会員証で特急券に引換えできないのですが?」
	\item [JR]「会員証で引換えできないですねー。おかしいな。クレジットカードでやってみましょう。駄目ですねー。」
	\item [客]「古いクレジットカードでは駄目ですか?」
	\item [JR]「あ、できましたね。はい、切符です。」
	\end{description}

\end{itemize} 


\section {モデル化の範囲}
この問題で登場する用語には以下がある。

\begin{itemize}
\item 鉄道会社A、鉄道会社B
\item おサイフケータイ
\item 特急券予約システム
\item 予約会員証、ICカード、予約カード
\item クレジットカード
	\begin{itemize}
	\item ロンロンViewカード、アトレクラブViewカード
	\end{itemize} 
\item 特急券
\end{itemize} 

「鉄道会社B」「ICカード」「予約カード」は、モデルを単純化するため対象としなかった。

「鉄道会社A」の「特急券予約システム」システムのうち、
以下の機能だけをVDM++\cite{SCSK2012PP}で要求仕様としてモデル化し、
問題点を明確化することにした。

\begin{itemize}
\item 予約する
\item 特急券を得る
\item クレジットカードを切り替える
\end{itemize} 

従って、モデル化する範囲に登場するpublicな用語は以下だけである。
\begin{itemize}
\item おサイフケータイ
\item 特急券予約システムシステム
\item 特急券予約システム
\item 予約会員証
\item クレジットカード
\item 特急券
\end{itemize} 





