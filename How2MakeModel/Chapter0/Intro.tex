形式手法で仕様を明確化する技術を教えている時に、
常に問われる「対象を如何にモデル化するか?」を説明するために作成したのが、本ブックレットである。

ここで言うモデル化とは、
システムの仕様に内部矛盾が無いかを検証(正当性検証)し、
ユーザーの要求に合っているかを確認する(妥当性確認)モデルを作成することである。

\section {対象者}
	\index{たいしょうしゃ@対象者}
	以下の対象者を想定して作成した。

	\begin{itemize}
	\item 現場において、システムの品質を改善し生産性を高めたいと考えているプロジェクトのリーダ、開発者の方
	\item 仕様作成とプログラム開発について、ある程度の知見を持っている方 
	\end{itemize}

\section {到達目標}
	\index{とうたつもくひょう@到達目標}

	到達目標は以下の通りである。

	\begin{itemize}
	\item 要求仕様の記述と分析の重要性が理解できていて、第三者に客観的に説明できる。 \\
		特に、厳密な仕様記述が品質向上と生産性向上に寄与することを理解している。
	\item 要求仕様作成のために、「対象を如何にモデル化するか」の手順とその意味が理解できている。
	\item 形式仕様記述言語VDM++を言語マニュアル\cite{Kyushu2016PP}等を参照しながら活用すれば、 \\
		小さな実システムの要求仕様を記述する事ができる。
	\end{itemize}

	上記の対象者と到達目標を考慮して、
	実際のシステム開発に流用することができ、
	読者にも知識がある問題について、
	小さな実例をベースに説明を行い、
	形式仕様記述言語VDM++のソースをできるだけ紹介することで、
	具体的イメージを持っていただくことにした。

	ただし、VDM++の言語仕様説明は必要最小限に留めたので、
	VDM++ソース部分を完全に理解するためには、
	VDM++言語マニュアル\cite{Kyushu2016PP}を参照していただきたい。

	本書のVDM++ソースは公開されている\footnote{\url{https://www.ipa.go.jp/files/000026828.zip}}ので、
	いつでもVDMのツールを用いて動かすことができる。
	ただし、今、この資料を読んでいる方は、IPAで公開されているものより新しい最新版の副読本と最新版のVDMTools
	\footnote{\url{http://fmvdm.org}}
	上で動くVDM++ソースを手に入れていることになるので、改めてIPAで公開されているものをダウンロードする必要はない。
	これがUMLなどの、レビュー以外は検証できず動かないモデル化と大きく異なる点である。

\section {本書の構成}
	\index{ほんしょのこうせい@本書の構成}

	本書は、I\hspace{-.1em}I部構成となっている。

	第\ref{MainBody}部では、対象を如何にモデル化するか?を説明する。

	\ref{WhyFormal}章でなぜ形式手法か?を紹介し、
	\ref{VDMabstract}章で形式手法VDMの概要を説明し、
	\ref{VDMeffect}章でVDM適用の成果を紹介する。

	そして、\ref{How2Intro}章でVDM導入・開発方法・体制の概要を示し、
	\ref{How2MakeModel}章で対象を如何にモデル化するか?を説明する。

	\ref{Conclusion}章でまとめを行い、\ref{Exercise}章で演習問題を示す。

	第\ref{ModelExamples}部では、モデルの具体例を紹介する。
	第\ref{MainBody}部の説明に関連するVDM++モデルのソースとその説明を行なっている。

	\ref{LibraryModel}章は、演習問題の解答例である図書館システムのモデルであり、
	実行不可能な陰仕様
		\footnote{関数・操作のインタフェースと、事前条件及び事後条件しか記述していない仕様を陰仕様と呼ぶ。
		陰仕様は、関数・操作の本体は記述していないが、「仕様」を記述していることになる。}
	と、それを回帰テストするための実行可能な陽仕様
		\footnote{関数・操作の本体が記述されていて実行可能な仕様を陽仕様と呼ぶ。}、
	さらに、やや設計仕様寄りのオブジェクト指向陽仕様としたモデルの例である。
	モデルの理解に必要な、最小限のVDM++文法も説明している。

	\ref{FareModel}章は本文中に出てくる運賃計算モデルであり、
	実システムの10分の1程度の行数でありながら、
	\ref{SpecFramework}節で説明する仕様フレームワークと
	\ref{RegressionTest}節で説明する回帰テストを使った要求仕様の例である。

	\ref{EvolvedExpressReservation}章は、\ref{How2MakeModel}章の
	特急券予約システムを改善したモデルの例であり、
	実用的なシステムを、ある目的のために抽象化して、小さな要求仕様モデルとして記述した例である。