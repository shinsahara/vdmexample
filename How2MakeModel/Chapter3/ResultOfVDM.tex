VDMの海外での適用事例は、
	\index{VDMのかいがいでのてきようじれい@VDMの海外での適用事例}
IBMでの多くのプログラミング言語開発、旧ソ連の100万行以上の仕様で記述された衛星システム、
VDM++からC++を生成したオランダの世界最大の花市場オークションシステムなどがある。

日本での適用事例には、表\ref{VDMeffectTable}がある。
	\index{VDMのにほんでの適用事例@VDMの日本での適用事例}

\begin{table}[h]
	\caption[VDMによる仕様作成の成果]{VDMによる仕様作成の成果}
	\index{VDMによるしようさくせいのこうか@VDMによる仕様作成の成果}
	\label{VDMeffectTable}
	\begin{center}
		\setlength{\tabcolsep}{3pt}
		\begin{tabular}{|c|c|c|r|r|r|r|} \hline
			開発組織 & 対象システム & 適用工程 & 仕様規模 & 欠陥数 & 生産性 & 開発期間   \\ \hline\hline
			日本フィッツ& 証券業務 & 実システム & 3+3万行 & 0 & 2.5倍 & 45\%  \\ 
			現SCSK &  & UseCase & 仕様+テストケース &  &  &   \\ 
			 &  & 要求仕様 &  &  &  &   \\ \hline
			フェリカネットワークス & おサイフケータイ &  実システム & 7.4+6.6万行 & 0 & 2倍 & 85\% \\ 
			 & ファームウェア &  API外部仕様 & 仕様+テストケース &  &  &   \\ \hline
			産業技術総合研究所 & 鉄道 & 既存システム & 0.75+80万行 &  従来仕様書の & &   \\ 
			オムロン & 駅務システム & 厳密仕様 & 仕様+テストケース & 欠陥発見数 & &   \\ 
			 &  & 作成実験 &  & (29) & &   \\ \hline
		\end{tabular}
	\end{center}
\end{table}

表\ref{VDMeffectTable}では、生産性と開発期間の比較は、
COCOMO81\cite{Boehm81}による見積りと実績値の比較で行った。
生産性のカラムは、見積もりに対して実績値が何倍になったかを示し、
開発期間のカラムは、見積もりに対して実績値が何\%であったかを示している。

仕様規模のカラムは、m + n という形式で、注釈抜きのVDM++ソース行数を表していて、
mの部分は仕様本体、nの部分はテストケースの行数を表している。

以下に、各システムの概要を述べる。

\section{日本フィッツの事例}
	\label{JFITS}
	\index{にほんふぃっつのじれい@日本フィッツの事例}

証券会社バックオフィスシステム
	\index{しょうけんがいしゃばっくおふぃすしすてむ@証券会社バックオフィスシステム}
用パッケージソフトウェアの2つのサブシステムにVDM++仕様を適用した。
最初に手がけたサブシステムは識別子に英語を使用したが、
2つ目のサブシステムでは、VDMToolsを修正して国際語化し、日本語識別子と日本語文字列を使用したVDM++仕様を作成した。
	\index{こくさいごか@国際語化}
	\index{にほんごしきべつし@日本語識別子}

開発チームは証券業務の知識がなく、仕様が曖昧なまま開発するのは危険と考え、
UseCaseレベルの要求仕様記述をVDM++を用いて行い、
回帰テストのテストケースもVDM++で記述し、
仕様の検証をVDMToolsの仕様アニメーション機能(シンボリック仕様実行)で行った。
回帰テストのコードカバレージをVDMToolsで計測し、
リリース時には95\%以上のカバレージを達成した。

開発チームのうちの2人がVDM++による記述を行ったが、
2人ともVDM++の知識はあったものの、実際にVDM++を業務として使うのは初めてだった。
プログラマもJavaとC++の知識はあったものの、実業務に使うのは初めてであった。

表\ref{VDMeffectTable}には、VDM++で開発した3000行あまりのライブラリの開発工数も含まれていて
、ライブラリを除くアプリケーション開発自体の生産性はさらに高いが、
ライブラリ開発とアプリケーション開発の工数を分離して計測する時間的余裕は無かったため、
ライブラリ開発がどの程度の割合を占めたかの正確なデータは残っていない。

\section{フェリカネットワークスの事例}
	\label{FeliCa}
	\index{ふぇりかねっとわーくすのじれい@フェリカネットワークスの事例}

以前のおサイフケータイ・ファームウェアの開発
	\index{おさいふけーたいふぁーむうぇあのかいはつ@おサイフケータイ・ファームウェアの開発}
で、仕様の曖昧さのため非常に苦労したため、
VDM++で仕様を書き、検証することになった。

日本フィッツの開発者をコンサルタントとし、
証券業務仕様の記述で用いた仕様記述フレームワーク
	\footnote{仕様記述フレームワークは、仕様のどの部分に何を書くかを定めた規則である。
	詳しくは、\ref{SpecFramework}節に説明する。}
を流用して、
予定通り開発を終了し、現在に至るまで欠陥が発生していない。

開発者に形式手法の知識はなかったため、
中心となる技術者にVDM++を教育し、
仕様記述フレームワークとその使用例を開発し、
それを元に、他の技術者が仕様を作成した。

API外部仕様は、今までどおりの日本語仕様と、VDM++仕様の両方を記述し、
矛盾する場合はVDM++仕様を優先することとした。

なお、本システムの対象はモバイルFeliCaチップであるが、
	\label{FeliCaちっぷ@FeliCaチップ}
本システムの後、EdyカードのFeliCaチップ開発にもVDM++が使用され、
	\label{Edyかーど@Edyカード}
本システムと同等以上の成果を上げた。
このシステムでは、従来型の日本語仕様は作成せず、
日本語識別子を使ったVDM++仕様のみを作成した。

また、セキュリティ強化版のFeliCaチップ
	\index{せきゅりてぃきょうかばんのFeliCaちっぷ@セキュリティ強化版のFeliCaチップ}
	\footnote{IC RC-SA00/1 SeriesとRC-SA00/2 Series}
	\label{FeliCaちっぷ@FeliCaチップ}
	\index{IC RC-SA00}
開発にもVDM++が使用され、
情報セキュリティ評価基準の国際標準であるコモンクライテリア(ISO/IEC 15408)において、
	\index{ISO/IEC 15408}
組み込みソフトウェアを搭載した非接触ICカードチップとして
	\index{ひせっしょくICかーどちっぷ@非接触ICカードチップ}
世界で初めて評価保証レベルEAL6+認証を取得した。
	\index{EAL6+}


\section{オムロンの事例}
	\label{OMRON}
	\index{おむろんのじれい@オムロンの事例}

駅務システム
	\index{えきむしすてむ@駅務システム}
は社会インフラとして高い信頼性が求められているが、
従来の仕様書は、今や、理解することや保守あるいは再利用が非常に困難となってきたため、
既存仕様をVDM++で書き換える実験を行った。

結果として、従来型仕様書の不備を29件発見し、
暗黙知
\footnote{ここで言う暗黙知は、担当者の一部だけが知っている業務知識で、明文化されていないものを指す。}
がVDM++で1400行相当発見され、
鉄道会社共通の仕様もVDM++で1400行発見された。

開発者は「暗黙知は仕様の欠陥であると認識した」と述べている。

検証は、産業技術総合研究所の検証サーバー「さつき」を利用し、
64コアCPUのマシンで実テストデータ80万件で5日間実行した。
このテストの実行速度は、実機の10分の1から20分の1であった
\footnote{SEC特別セミナーアーキテクチャ指向エンジニアリングと形式手法における以下の発表資料による。
幡山五郎、大崎人士、相馬大輔、Nguyen Van Tang 著、
上流工程大規模テストのための技術開発
~組込みシステム開発のフロントローディング化~,
2011年7月5日
}。




