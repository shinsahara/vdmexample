\documentclass{jarticle}
\usepackage{twocolumn}
\usepackage{sea-ss}

\usepackage{url}

%VDM関係のpackage
\usepackage{color}
\usepackage{vpp-nms}
\usepackage{longtable}
%--------------------------------

\title{ソフトウェアシンポジウムの投稿要領について}
\author{荒木 啓二郎\\ 熊本高等専門学校\\
{\ttfamily ss2019inquiry@sea.jp} \and
富松 篤典\\ 電盛社\\ 
{\ttfamily ss2019inquiry@sea.jp}
}

%カメラレディでは以下の行を有効にしてください.
%\pagestyle{empty}

\begin{document}
\sloppy
\maketitle

%カメラレディでは以下の行を有効にしてください.
%\thispagestyle{empty} 

\begin{abstract}
このテンプレートはソフトウェアシンポジウム用に作られたものです.

ここには要旨を書いてください.
\end{abstract}

\Section{はじめに}
本文は二段組にしてください.

うまくスタイルファイルが使えない場合は,
\url{http://sea.jp/ss2019/paper_submission.php}
に用意した組版結果にしたがって書いてください.

\Section{文書体裁に付いて}


%\section {図書館1ビジネスロジック}
\include{Library1.vpp}
%\section {図書館1要求辞書}
\include{LibraryRQ1.vdmpp}


\SubSection{段落について}

一文字段下げします.
昨年までの指示とは異なりますので注意してください.

\SubSection{句読点について}
句読点には半角「,」「.」ではなく全角「,」「.」を使用してください.
昨年までの指示とは異なりますので注意してください.

\SubSection{見出しについて}

セクションの見出しの番号に.をつけるために,
\verb|\section|と\verb|\subsection|の代わりに
\verb|\Section|と\verb|\SubSection|を用いてください.

\SubSection{著者所属について}
著者所属は忘れずにお書きください.
住所,電話番号等については書きたい方は書いてください.

\Section{おわりに}

参考文献の書き方はこのソースファイルの最後を参考にしてください.
これはあくまでも例ですが,
DesignPattern\cite{Cond95}のように参照できます.
質問は\url{ss2019inquiry @ sea.jp}まで宜しくお願いします.

\begin{thebibliography}{9}

\bibitem{Cond95}
Huni,~H., R.~Johnson, and R.~Engel ``A Framework for Network Protocol Software'',
{\em Proceedings of OOPSLA'95}, pp. 358--369, 1995.
\end{thebibliography}
\end{document}
