\documentclass[a4paper,10pt]{jsarticle}
%\documentclass[a4paper,10pt]{jsbook}

\usepackage[dvipdfmx]{graphicx, color}
%\usepackage{folha}
\graphicspath{{image/}}

\usepackage{color}
\usepackage{array}
\usepackage{longtable}
\usepackage{alltt}
\usepackage{graphics}
\usepackage{vpp-nms}
%\usepackage{vpp}
\usepackage{makeidx}
\makeindex

\usepackage{colortbl}

\usepackage[dvipdfmx,bookmarks=true,bookmarksnumbered=true,colorlinks,plainpages=true]{hyperref}

%\AtBeginDvi{\special{pdf:tounicode 90ms-RKSJ-UCS2}}
\AtBeginDvi{\special{pdf:tounicode EUC-UCS2}}

\definecolor{covered}{rgb}{0,0,0}      %black
\definecolor{not-covered}{rgb}{1,0,0}  %red

\setcounter{secnumdepth}{6}
\makeatletter
\renewcommand{\paragraph}{\@startsection{paragraph}{4}{\z@}%
  {1.5\Cvs \@plus.5\Cdp \@minus.2\Cdp}%
  {.5\Cvs \@plus.3\Cdp}%
  {\reset@font\normalsize\bfseries}}
\makeatother

\renewcommand{\sf}{\sffamily \color{blue}}

\newcommand{\syou}{\texttt{<}}
\newcommand{\dai}{\texttt{>}}

%\include{Title}

%\pagestyle{empty}
\usepackage{fancyhdr}
%\usepackage{lastpage} 
%  \pagestyle{fancy} 
%   \let\origtitle\title 
%  \renewcommand{\title}[1]{\lfoot{#1}\origtitle{#1}}

%  \rfoot{\today}
%  \rhead{[\ \scshape\oldstylenums{\thepage}\ / %
 %     \scshape\oldstylenums{\pageref{LastPage}}\ ]}
%  \cfoot{}


\begin{document}

% the title page

\title{VDM++言語入門セミナー 演習問題\\
\small{($Revision: 0.3 $ -- \today)}}
\author{佐原 伸\\
}
%\institute{\pgldk \and \chessnl}
\date{\mbox{}}
\maketitle

\begin{abstract}
\setlength{\baselineskip}{12pt plus .1pt}
%VDM++言語\cite{CSK2007PP}入門セミナーの演習問題である。
「対象を如何にモデル化するか?」のモデル作成例。
\end{abstract}
%\vspace{-1cm}

\tableofcontents

%\mainmatter
%\section {写像のCRUDライブラリー}
\include{CRUD_map.vdmpp}
%\section {図書館1ビジネスロジック}
\include{Library1.vpp}
%\section {図書館1要求辞書}
\include{LibraryRQ1.vdmpp}
%\section {図書館1回帰テストケース}
\include{MyCRUDlib_TestCases.vdmpp}
%\include{MyTest.vpp}
%\include{MyTestCase.vpp}

%\section {ユーティリティ SSlib}
%\include{Character.vpp}
%\include{Sequence.vpp}
%\include{String.vpp}

\newpage
%\layout

%\begin{thebibliography}{9}
\section{参考文献、索引}
VDM++\cite{Kyushu2016PP}は、
1970年代中頃にIBMウィーン研究所で開発されたVDM-SL\cite{Kyushu2016SL}を拡張し、
さらにオブジェクト指向拡張した形式仕様記述言語である。


VDM++の教科書としては\cite{Sakoh2010}がある。

VDM++を開発現場で実践的に使う場合の解説が\cite{Sahara2008}にある。

\bibliographystyle{jplain}
%\bibliography{/Users/sahara/svnw/sahara}
\bibliography{/Users/sahara/Dropbox/bib/saharaUTF8}
%\bibliography{/Users/ssahara/svnwork/sahara}

%\end{thebibliography}

%\newpage
%\addcontentsline{toc}{section}{Index}
\printindex

\end{document}
