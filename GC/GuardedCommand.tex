\documentclass[a4paper,10pt]{jsarticle}
%\documentclass[a4paper,10pt]{jsbook}

\usepackage[dvipdfmx]{graphicx, color}
%\usepackage{folha}
\graphicspath{{image/}}

\usepackage{color}
\usepackage{array}
\usepackage{longtable}
\usepackage{alltt}
\usepackage{graphics}
\usepackage{vpp-nms}
%\usepackage{vpp}
\usepackage{makeidx}
\makeindex

\usepackage{colortbl}

\usepackage[dvipdfmx,bookmarks=true,bookmarksnumbered=true,colorlinks,plainpages=true]{hyperref}

%\AtBeginDvi{\special{pdf:tounicode 90ms-RKSJ-UCS2}}
\AtBeginDvi{\special{pdf:tounicode EUC-UCS2}}

\definecolor{covered}{rgb}{0,0,0}      %black
\definecolor{not-covered}{rgb}{1,0,0}  %red

\setcounter{secnumdepth}{6}
\makeatletter
\renewcommand{\paragraph}{\@startsection{paragraph}{4}{\z@}%
  {1.5\Cvs \@plus.5\Cdp \@minus.2\Cdp}%
  {.5\Cvs \@plus.3\Cdp}%
  {\reset@font\normalsize\bfseries}}
\makeatother

\renewcommand{\sf}{\sffamily \color{blue}}

\newcommand{\syou}{\texttt{<}}
\newcommand{\dai}{\texttt{>}}

%\include{Title}

%\pagestyle{empty}
\usepackage{fancyhdr}
\usepackage{lastpage} 
  \pagestyle{fancy} 
   \let\origtitle\title 
  \renewcommand{\title}[1]{\lfoot{#1}\origtitle{#1}}

  \rfoot{\today}
  \rhead{[\ \scshape\oldstylenums{\thepage}\ / %
      \scshape\oldstylenums{\pageref{LastPage}}\ ]}
  \cfoot{}


\begin{document}

% the title page
\title{入退出管理システムVDM++仕様}
\author{佐原 伸\\\\
タオベアーズ\\
}
%\institute{\pgldk \and \chessnl}
\date{\mbox{}}
\maketitle

%\TaoReport{ガードコマンド・モデル}{\today}{タオベアーズ}{佐原伸}
%\setlength{\baselineskip}{12pt plus .1pt}
%\tolerance 10000
\tableofcontents
%\thispagestyle{empty} 

%\include{Abstract}
\include{BlackBoard.vpp}
\include{Common.vpp}
\include{Door.vpp}
\include{Door_Real_Enter.vpp}
\include{Door_Real_Exit.vpp}
\include{Door_Vairtual.vpp}
\documentclass[a4paper,10pt]{jsarticle}
%\documentclass[a4paper,10pt]{jsbook}

\usepackage[dvipdfmx]{graphicx, color}
%\usepackage{folha}
\graphicspath{{image/}}

\usepackage{color}
\usepackage{array}
\usepackage{longtable}
\usepackage{alltt}
\usepackage{graphics}
\usepackage{vpp-nms}
%\usepackage{vpp}
\usepackage{makeidx}
\makeindex

\usepackage{colortbl}

\usepackage[dvipdfmx,bookmarks=true,bookmarksnumbered=true,colorlinks,plainpages=true]{hyperref}

%\AtBeginDvi{\special{pdf:tounicode 90ms-RKSJ-UCS2}}
\AtBeginDvi{\special{pdf:tounicode EUC-UCS2}}

\definecolor{covered}{rgb}{0,0,0}      %black
\definecolor{not-covered}{rgb}{1,0,0}  %red

\setcounter{secnumdepth}{6}
\makeatletter
\renewcommand{\paragraph}{\@startsection{paragraph}{4}{\z@}%
  {1.5\Cvs \@plus.5\Cdp \@minus.2\Cdp}%
  {.5\Cvs \@plus.3\Cdp}%
  {\reset@font\normalsize\bfseries}}
\makeatother

\renewcommand{\sf}{\sffamily \color{blue}}

\newcommand{\syou}{\texttt{<}}
\newcommand{\dai}{\texttt{>}}

%\include{Title}

%\pagestyle{empty}
\usepackage{fancyhdr}
\usepackage{lastpage} 
  \pagestyle{fancy} 
   \let\origtitle\title 
  \renewcommand{\title}[1]{\lfoot{#1}\origtitle{#1}}

  \rfoot{\today}
  \rhead{[\ \scshape\oldstylenums{\thepage}\ / %
      \scshape\oldstylenums{\pageref{LastPage}}\ ]}
  \cfoot{}


\begin{document}

% the title page
\title{入退出管理システムVDM++仕様}
\author{佐原 伸\\\\
タオベアーズ\\
}
%\institute{\pgldk \and \chessnl}
\date{\mbox{}}
\maketitle

%\TaoReport{ガードコマンド・モデル}{\today}{タオベアーズ}{佐原伸}
%\setlength{\baselineskip}{12pt plus .1pt}
%\tolerance 10000
\tableofcontents
%\thispagestyle{empty} 

%\include{Abstract}
\include{BlackBoard.vpp}
\include{Common.vpp}
\include{Door.vpp}
\include{Door_Real_Enter.vpp}
\include{Door_Real_Exit.vpp}
\include{Door_Vairtual.vpp}
\documentclass[a4paper,10pt]{jsarticle}
%\documentclass[a4paper,10pt]{jsbook}

\usepackage[dvipdfmx]{graphicx, color}
%\usepackage{folha}
\graphicspath{{image/}}

\usepackage{color}
\usepackage{array}
\usepackage{longtable}
\usepackage{alltt}
\usepackage{graphics}
\usepackage{vpp-nms}
%\usepackage{vpp}
\usepackage{makeidx}
\makeindex

\usepackage{colortbl}

\usepackage[dvipdfmx,bookmarks=true,bookmarksnumbered=true,colorlinks,plainpages=true]{hyperref}

%\AtBeginDvi{\special{pdf:tounicode 90ms-RKSJ-UCS2}}
\AtBeginDvi{\special{pdf:tounicode EUC-UCS2}}

\definecolor{covered}{rgb}{0,0,0}      %black
\definecolor{not-covered}{rgb}{1,0,0}  %red

\setcounter{secnumdepth}{6}
\makeatletter
\renewcommand{\paragraph}{\@startsection{paragraph}{4}{\z@}%
  {1.5\Cvs \@plus.5\Cdp \@minus.2\Cdp}%
  {.5\Cvs \@plus.3\Cdp}%
  {\reset@font\normalsize\bfseries}}
\makeatother

\renewcommand{\sf}{\sffamily \color{blue}}

\newcommand{\syou}{\texttt{<}}
\newcommand{\dai}{\texttt{>}}

%\include{Title}

%\pagestyle{empty}
\usepackage{fancyhdr}
\usepackage{lastpage} 
  \pagestyle{fancy} 
   \let\origtitle\title 
  \renewcommand{\title}[1]{\lfoot{#1}\origtitle{#1}}

  \rfoot{\today}
  \rhead{[\ \scshape\oldstylenums{\thepage}\ / %
      \scshape\oldstylenums{\pageref{LastPage}}\ ]}
  \cfoot{}


\begin{document}

% the title page
\title{入退出管理システムVDM++仕様}
\author{佐原 伸\\\\
タオベアーズ\\
}
%\institute{\pgldk \and \chessnl}
\date{\mbox{}}
\maketitle

%\TaoReport{ガードコマンド・モデル}{\today}{タオベアーズ}{佐原伸}
%\setlength{\baselineskip}{12pt plus .1pt}
%\tolerance 10000
\tableofcontents
%\thispagestyle{empty} 

%\include{Abstract}
\include{BlackBoard.vpp}
\include{Common.vpp}
\include{Door.vpp}
\include{Door_Real_Enter.vpp}
\include{Door_Real_Exit.vpp}
\include{Door_Vairtual.vpp}
\documentclass[a4paper,10pt]{jsarticle}
%\documentclass[a4paper,10pt]{jsbook}

\usepackage[dvipdfmx]{graphicx, color}
%\usepackage{folha}
\graphicspath{{image/}}

\usepackage{color}
\usepackage{array}
\usepackage{longtable}
\usepackage{alltt}
\usepackage{graphics}
\usepackage{vpp-nms}
%\usepackage{vpp}
\usepackage{makeidx}
\makeindex

\usepackage{colortbl}

\usepackage[dvipdfmx,bookmarks=true,bookmarksnumbered=true,colorlinks,plainpages=true]{hyperref}

%\AtBeginDvi{\special{pdf:tounicode 90ms-RKSJ-UCS2}}
\AtBeginDvi{\special{pdf:tounicode EUC-UCS2}}

\definecolor{covered}{rgb}{0,0,0}      %black
\definecolor{not-covered}{rgb}{1,0,0}  %red

\setcounter{secnumdepth}{6}
\makeatletter
\renewcommand{\paragraph}{\@startsection{paragraph}{4}{\z@}%
  {1.5\Cvs \@plus.5\Cdp \@minus.2\Cdp}%
  {.5\Cvs \@plus.3\Cdp}%
  {\reset@font\normalsize\bfseries}}
\makeatother

\renewcommand{\sf}{\sffamily \color{blue}}

\newcommand{\syou}{\texttt{<}}
\newcommand{\dai}{\texttt{>}}

%\include{Title}

%\pagestyle{empty}
\usepackage{fancyhdr}
\usepackage{lastpage} 
  \pagestyle{fancy} 
   \let\origtitle\title 
  \renewcommand{\title}[1]{\lfoot{#1}\origtitle{#1}}

  \rfoot{\today}
  \rhead{[\ \scshape\oldstylenums{\thepage}\ / %
      \scshape\oldstylenums{\pageref{LastPage}}\ ]}
  \cfoot{}


\begin{document}

% the title page
\title{入退出管理システムVDM++仕様}
\author{佐原 伸\\\\
タオベアーズ\\
}
%\institute{\pgldk \and \chessnl}
\date{\mbox{}}
\maketitle

%\TaoReport{ガードコマンド・モデル}{\today}{タオベアーズ}{佐原伸}
%\setlength{\baselineskip}{12pt plus .1pt}
%\tolerance 10000
\tableofcontents
%\thispagestyle{empty} 

%\include{Abstract}
\include{BlackBoard.vpp}
\include{Common.vpp}
\include{Door.vpp}
\include{Door_Real_Enter.vpp}
\include{Door_Real_Exit.vpp}
\include{Door_Vairtual.vpp}
\include{GuardedCommand.vpp}
\include{GuardedCommand_Door.vpp}
\include{GuardedCommand_Guide.vpp}
\include{GuardedCommand_IDTerminal.vpp}
\include{GuardedCommand_IDTerminal_OK.vpp}
\include{GuardedCommand_IDTerminal_OK_Virtual.vpp}
\include{GuardedCommand_IDTerminal_NG.vpp}
\include{GuardedCommand_IDTerminal_Gamma_OK.vpp}
\include{GuardedCommand_IDTerminal_Gamma_OK_Virtual.vpp}
\include{GuardedCommand_IDTerminal_Gamma_NG.vpp}
\include{GuardedCommand_Sensor.vpp}
\include{GuardedCommand_Sensor_Virtual.vpp}
\include{GuardedCommand_User.vpp}
\include{GuardedCommand_UserCertified.vpp}
\include{GuardedCommand_UserCertified_Virtual.vpp }
\include{GuardedCommand_UserDoorOpen.vpp}
\include{GuardedCommand_UserDoorClose.vpp}
\include{GuardedCommand_UserGuidedMove.vpp}
\include{GuardedCommand_UserMove.vpp}
\include{Guide.vpp}
\include{IDCard.vpp}
\include{IDTerminal.vpp}
\include{LocationGraph.vpp}
\include{Object.vpp}
\include{Scheduler.vpp}
\include{Sensor.vpp}
\include{TestApp.vpp}
\include{TestCase.vpp}
\include{TouchPanel.vpp}
\include{User.vpp}

%\begin{thebibliography}{9}
\section{参考文献等}
VDM++\cite{Kyushu2016PP}は、
1970年代中頃にIBMウィーン研究所で開発されたVDM-SL\cite{Kyushu2016SL}を拡張し、
さらにオブジェクト指向拡張した形式仕様記述言語で、フリーソフトウェア\footnote{GNU General Public License v3.0}である。
\bibliographystyle{jplain}
%\bibliography{/Users/sahara/svnw/sahara}
\bibliography{/Users/sahara/Dropbox/bib/saharaUTF8}
%\bibliography{/Users/ssahara/svnwork/sahara}

%\end{thebibliography}

%\newpage
%\addcontentsline{toc}{section}{Index}
\printindex

\end{document}

\include{GuardedCommand_Door.vpp}
\include{GuardedCommand_Guide.vpp}
\include{GuardedCommand_IDTerminal.vpp}
\include{GuardedCommand_IDTerminal_OK.vpp}
\include{GuardedCommand_IDTerminal_OK_Virtual.vpp}
\include{GuardedCommand_IDTerminal_NG.vpp}
\include{GuardedCommand_IDTerminal_Gamma_OK.vpp}
\include{GuardedCommand_IDTerminal_Gamma_OK_Virtual.vpp}
\include{GuardedCommand_IDTerminal_Gamma_NG.vpp}
\include{GuardedCommand_Sensor.vpp}
\include{GuardedCommand_Sensor_Virtual.vpp}
\include{GuardedCommand_User.vpp}
\include{GuardedCommand_UserCertified.vpp}
\include{GuardedCommand_UserCertified_Virtual.vpp }
\include{GuardedCommand_UserDoorOpen.vpp}
\include{GuardedCommand_UserDoorClose.vpp}
\include{GuardedCommand_UserGuidedMove.vpp}
\include{GuardedCommand_UserMove.vpp}
\include{Guide.vpp}
\include{IDCard.vpp}
\include{IDTerminal.vpp}
\include{LocationGraph.vpp}
\include{Object.vpp}
\include{Scheduler.vpp}
\include{Sensor.vpp}
\include{TestApp.vpp}
\include{TestCase.vpp}
\include{TouchPanel.vpp}
\include{User.vpp}

%\begin{thebibliography}{9}
\section{参考文献等}
VDM++\cite{Kyushu2016PP}は、
1970年代中頃にIBMウィーン研究所で開発されたVDM-SL\cite{Kyushu2016SL}を拡張し、
さらにオブジェクト指向拡張した形式仕様記述言語で、フリーソフトウェア\footnote{GNU General Public License v3.0}である。
\bibliographystyle{jplain}
%\bibliography{/Users/sahara/svnw/sahara}
\bibliography{/Users/sahara/Dropbox/bib/saharaUTF8}
%\bibliography{/Users/ssahara/svnwork/sahara}

%\end{thebibliography}

%\newpage
%\addcontentsline{toc}{section}{Index}
\printindex

\end{document}

\include{GuardedCommand_Door.vpp}
\include{GuardedCommand_Guide.vpp}
\include{GuardedCommand_IDTerminal.vpp}
\include{GuardedCommand_IDTerminal_OK.vpp}
\include{GuardedCommand_IDTerminal_OK_Virtual.vpp}
\include{GuardedCommand_IDTerminal_NG.vpp}
\include{GuardedCommand_IDTerminal_Gamma_OK.vpp}
\include{GuardedCommand_IDTerminal_Gamma_OK_Virtual.vpp}
\include{GuardedCommand_IDTerminal_Gamma_NG.vpp}
\include{GuardedCommand_Sensor.vpp}
\include{GuardedCommand_Sensor_Virtual.vpp}
\include{GuardedCommand_User.vpp}
\include{GuardedCommand_UserCertified.vpp}
\include{GuardedCommand_UserCertified_Virtual.vpp }
\include{GuardedCommand_UserDoorOpen.vpp}
\include{GuardedCommand_UserDoorClose.vpp}
\include{GuardedCommand_UserGuidedMove.vpp}
\include{GuardedCommand_UserMove.vpp}
\include{Guide.vpp}
\include{IDCard.vpp}
\include{IDTerminal.vpp}
\include{LocationGraph.vpp}
\include{Object.vpp}
\include{Scheduler.vpp}
\include{Sensor.vpp}
\include{TestApp.vpp}
\include{TestCase.vpp}
\include{TouchPanel.vpp}
\include{User.vpp}

%\begin{thebibliography}{9}
\section{参考文献等}
VDM++\cite{Kyushu2016PP}は、
1970年代中頃にIBMウィーン研究所で開発されたVDM-SL\cite{Kyushu2016SL}を拡張し、
さらにオブジェクト指向拡張した形式仕様記述言語で、フリーソフトウェア\footnote{GNU General Public License v3.0}である。
\bibliographystyle{jplain}
%\bibliography{/Users/sahara/svnw/sahara}
\bibliography{/Users/sahara/Dropbox/bib/saharaUTF8}
%\bibliography{/Users/ssahara/svnwork/sahara}

%\end{thebibliography}

%\newpage
%\addcontentsline{toc}{section}{Index}
\printindex

\end{document}

\include{GuardedCommand_Door.vpp}
\include{GuardedCommand_Guide.vpp}
\include{GuardedCommand_IDTerminal.vpp}
\include{GuardedCommand_IDTerminal_OK.vpp}
\include{GuardedCommand_IDTerminal_OK_Virtual.vpp}
\include{GuardedCommand_IDTerminal_NG.vpp}
\include{GuardedCommand_IDTerminal_Gamma_OK.vpp}
\include{GuardedCommand_IDTerminal_Gamma_OK_Virtual.vpp}
\include{GuardedCommand_IDTerminal_Gamma_NG.vpp}
\include{GuardedCommand_Sensor.vpp}
\include{GuardedCommand_Sensor_Virtual.vpp}
\include{GuardedCommand_User.vpp}
\include{GuardedCommand_UserCertified.vpp}
\include{GuardedCommand_UserCertified_Virtual.vpp }
\include{GuardedCommand_UserDoorOpen.vpp}
\include{GuardedCommand_UserDoorClose.vpp}
\include{GuardedCommand_UserGuidedMove.vpp}
\include{GuardedCommand_UserMove.vpp}
\include{Guide.vpp}
\include{IDCard.vpp}
\include{IDTerminal.vpp}
\include{LocationGraph.vpp}
\include{Object.vpp}
\include{Scheduler.vpp}
\include{Sensor.vpp}
\include{TestApp.vpp}
\include{TestCase.vpp}
\include{TouchPanel.vpp}
\include{User.vpp}

%\begin{thebibliography}{9}
\section{参考文献等}
VDM++\cite{Kyushu2016PP}は、
1970年代中頃にIBMウィーン研究所で開発されたVDM-SL\cite{Kyushu2016SL}を拡張し、
さらにオブジェクト指向拡張した形式仕様記述言語で、フリーソフトウェア\footnote{GNU General Public License v3.0}である。
\bibliographystyle{jplain}
%\bibliography{/Users/sahara/svnw/sahara}
\bibliography{/Users/sahara/Dropbox/bib/saharaUTF8}
%\bibliography{/Users/ssahara/svnwork/sahara}

%\end{thebibliography}

%\newpage
%\addcontentsline{toc}{section}{Index}
\printindex

\end{document}
