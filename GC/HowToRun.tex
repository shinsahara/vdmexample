\documentclass[a4paper,8pt]{jsarticle}
%\documentclass[a4paper,10pt]{jsbook}

\usepackage[dvipdfmx]{graphicx, color}
%\usepackage{folha}
\graphicspath{{image/}}

\usepackage{color}
\usepackage{array}
\usepackage{longtable}
\usepackage{alltt}
\usepackage{graphics}
\usepackage{vpp-nms}
%\usepackage{vpp}
\usepackage{makeidx}

\usepackage{colortbl}

\usepackage[dvipdfmx,bookmarks=true,bookmarksnumbered=true,colorlinks,plainpages=true]{hyperref}

%\AtBeginDvi{\special{pdf:tounicode 90ms-RKSJ-UCS2}}
\AtBeginDvi{\special{pdf:tounicode EUC-UCS2}}


\definecolor{covered}{rgb}{0,0,0}      %black
\definecolor{not-covered}{rgb}{1,0,0}  %red

\setcounter{secnumdepth}{6}
\makeatletter
\renewcommand{\paragraph}{\@startsection{paragraph}{4}{\z@}%
  {1.5\Cvs \@plus.5\Cdp \@minus.2\Cdp}%
  {.5\Cvs \@plus.3\Cdp}%
  {\reset@font\normalsize\bfseries}}
\makeatother

\renewcommand{\sf}{\sffamily \color{blue}}

\newcommand{\syou}{\texttt{<}}
\newcommand{\dai}{\texttt{>}}

%\include{Title}

%\pagestyle{empty}
\usepackage{fancyhdr}
\usepackage{lastpage} 
  \pagestyle{fancy} 
   \let\origtitle\title 
  \renewcommand{\title}[1]{\lfoot{#1}\origtitle{#1}}

  \rfoot{\today}
  \rhead{[\ \scshape\oldstylenums{\thepage}\ / %
      \scshape\oldstylenums{\pageref{LastPage}}\ ]}
  \cfoot{}


\begin{document}

% the title page
\title{入退出管理システム:モデルの動かし方}
\author{佐原 伸\\\\
タオベアーズ\\
}
%\institute{\pgldk \and \chessnl}
\date{\mbox{}}
\maketitle

%\TaoReport{ガードコマンド・モデル}{\today}{タオベアーズ}{佐原伸}
%\setlength{\baselineskip}{12pt plus .1pt}
%\tolerance 10000
\tableofcontents
%\thispagestyle{empty} 

\newpage

\section {VDMTools GUI版による動かし方}
\label{runFromGUI}

以下の手順に従って、モデルを動かす。
なお、VDMToolsの起動方法自体は、ユーザーズマニュアル\cite{VDMManual}を参照のこと。

\begin{enumerate}
	\item vppgdeをダブルクリックして、VDMTools GUI版を起動する
	\item VDMToolsのプロジェクトファイル gc.prj を読み込む
	\item i とタイプしてインタープリタを初期化する
	\item debug new T().r() とタイプして、回帰テストを起動する
\end{enumerate}

以下のようにモデルが動く。

\begin{verbatim}
>>  i
Initializing specification ... done
done
>> debug new T().r()
Testing ...<<TestCaseST0010:利用者の一人が場所5で認証されていない。>>
利用者が、案内表示を作成しました。mk_( objref2386(利用者):
  < オブジェクト`s現在時間(S) = 100,
    共通定義`debug(S) = 5,
    利用者`sIDカード = objref79,
    利用者`s作成者名 = [  ],
    利用者`s利用者ID = 3,
    利用者`s到着時間 = 250,
    利用者`s前にいた場所 = nil,
    利用者`s躊躇する最大時間 = 75 >, 2, objref2387(案内表示):
  < オブジェクト`s現在時間(S) = 100,
    共通定義`debug(S) = 5,
    案内表示`sガイドする場所ID = 4,
    案内表示`s案内する利用者 = objref2386,
    案内表示`s案内表示ID = 2 > )
...中略
Parsing "BlackBoard.vpp" (Latex) ... done
Parsing "GardedCommand_UserDoorClose.vpp" (Latex) ... done
Parsing "Common.vpp" (Latex) ... done
Parsing "GardedCommand_UserDoorOpen.vpp" (Latex) ... done
Parsing "Door.vpp" (Latex) ... done
Parsing "GardedCommand_UserGuidedMove.vpp" (Latex) ... done
Parsing "Door_Real_Enter.vpp" (Latex) ... done
Parsing "GardedCommand_UserMove.vpp" (Latex) ... done

...中略
利用者が、移動しました。mk_( objref3600(利用者):
  < オブジェクト`s現在時間(S) = 125,
    共通定義`debug(S) = 5,
    利用者`sIDカード = objref2183,
    利用者`s作成者名 = "ガードコマンド_利用者案内無し移動",
    利用者`s利用者ID = 1,
    利用者`s到着時間 = 125,
    利用者`s前にいた場所 = 0,
    利用者`s躊躇する最大時間 = 10 >, 0, 1 )

現在時間=125のすべてのガードコマンドupdateを完了しました。

現在時間=150のすべてのガードコマンドupdateを完了しました。

現在時間=175のすべてのガードコマンドupdateを完了しました。
      Done <<TestCaseST0001:案内無しに利用者が場所0から1へ移動する>>.
 
*** All Tests Passed. ***
(no return value)
\end{verbatim}

\section {VDMTools コマンド版による動かし方}
\label{runFromCygwin}
VDMTools コマンド版を、下記の手順で、
cygwin
\footnote{http://cygwin.com/cygwin-ug-net/cygwin-ug-net.html}
からmakeコマンドで動かすことにより、
回帰テストを簡単に行うことができる。

\begin{enumerate}
	\item cygwinを起動する
	\item モデルのあるフォルダーに移る
	\item make test とタイプして、回帰テストを起動する
\end{enumerate}

以下のようにモデルが動く。

\begin{verbatim}
$ make test
...中略
Parsing "BlackBoard.vpp" (Latex) ... done
Parsing "GardedCommand_UserDoorClose.vpp" (Latex) ... done
Parsing "Common.vpp" (Latex) ... done
Parsing "GardedCommand_UserDoorOpen.vpp" (Latex) ... done
Parsing "Door.vpp" (Latex) ... done
Parsing "GardedCommand_UserGuidedMove.vpp" (Latex) ... done
Parsing "Door_Real_Enter.vpp" (Latex) ... done
Parsing "GardedCommand_UserMove.vpp" (Latex) ... done

...中略
利用者が、移動しました。mk_( objref3600(利用者):
  < オブジェクト`s現在時間(S) = 125,
    共通定義`debug(S) = 5,
    利用者`sIDカード = objref2183,
    利用者`s作成者名 = "ガードコマンド_利用者案内無し移動",
    利用者`s利用者ID = 1,
    利用者`s到着時間 = 125,
    利用者`s前にいた場所 = 0,
    利用者`s躊躇する最大時間 = 10 >, 0, 1 )

現在時間=125のすべてのガードコマンドupdateを完了しました。

現在時間=150のすべてのガードコマンドupdateを完了しました。

現在時間=175のすべてのガードコマンドupdateを完了しました。
      Done <<TestCaseST0001:案内無しに利用者が場所0から1へ移動する>>.
 
*** All Tests Passed. ***
(no return value)
\end{verbatim}

\section {ドキュメントの作成方法}
\label{makeDoc}
本ドキュメント、モデルの考察、および注釈付きソースコードの各ドキュメントは、
下記の手順で\ref{runFromCygwin}節と同様に簡単に作成できる。

\begin{enumerate}
	\item cygwinを起動する
	\item モデルのあるフォルダーに移る
	\item make all とタイプする
\end{enumerate}

なお、make allを行った場合、vppdeコマンドが見つからないというエラーが出た場合は、
vppdeコマンドのPATH設定が行われていない可能性があるので、下記のように
ホーム・ディレクトリの .bashrc ファイルにPATH設定を行う。

\begin{verbatim}
export PATH=/cygdrive/c/vpp/bin/:$PATH
\end{verbatim}

Windows 7では、最初にvppdeを起動した時、vppdeが何の応答も返さなくなることがあるが、
下記のように、一時ファイル・ディレクトリを .bashrc ファイルに指定することで、動くはずである。

\begin{verbatim}
export TMPDIR=C:/cygwin/tmp
\end{verbatim}

また、 I didn't find a database entry for "VDMManual"といったエラーメッセージが出た時は、
HowToRun.texファイルの下記箇所を、納入されたファイルを格納したディレクトリに変更すれば、
参考文献が正しくドキュメント上に表示される。

\begin{verbatim}
\bibliography{/Users/sahara/bib/sahara}
\end{verbatim}

%\begin{thebibliography}{9}
\section{参考文献等}

\bibliographystyle{jplain}
%\bibliography{/Users/sahara/svnw/sahara}
\bibliography{/Users/sahara/Dropbox/bib/saharaUTF8}

%\end{thebibliography}

%\newpage
%\addcontentsline{toc}{section}{Index}
%\printindex

\end{document}
